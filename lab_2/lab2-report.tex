% arara: indent: {overwrite: yes}
\section{Лабораторная 2}

\subsection{Условие}

\textbf{Согласно варианту 10:}
\begin{enumerate}
	\item Пуассона – $\Pi(\lambda), \lambda = 0.7$; Геометрическое – $G(p), p = 0.2$;
	\item Бернулли – $Bi(1, p), p = 0.75$; Пуассона – $\Pi(\lambda), \lambda = 1$;
\end{enumerate}

Смоделировать дискретную случайную величину. Исследовать точность моделирования.

\begin{enumerate}
	\item Осуществить моделирование $n = 1000$ реализаций СВ из заданных дискретных распределений;
	\item Вывести на экран несмещённые оценки математического ожидания и дисперсии, сравнить их с истинными значениями;
	\item Для каждой из случайных величин построить свой $\chi^{2}$-критйрий Пирсона с уровнем значимости $\varepsilon = 0.05$. Проверить, что вероятность ошибки I рода стремится к 0.05;
	\item Осуществить проверку каждой из сгенерированных выборок каждым из построенных критериев.
\end{enumerate}
