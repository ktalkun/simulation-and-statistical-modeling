% arara: indent: {overwrite: yes}
\section{Лабораторная 6}

\subsection{Условие}

\textbf{Согласно варианту 10:}
Моделирование процесса функционирования вычислительного центра.\\
\\
\textbf{Исходные данные:}
\begin{enumerate}
	\item Вычислительный центр, оснащенный тремя однотипными ЭВМ, обслуживает сеть активных терминалов;
	\item Задачи пользователей образуют пуассоновский поток с $\lambda$ зад/сек, а время выполнения задачи в ЭВМ имеет экспоненциальное распределение с математическим ожиданием $\mu$ сек;
	\item Программа-диспетчер обрабатывает задачу, выбирая для нее свободную ЭВМ. Время обработки равномерно распределено на интервале $[a\pm\delta]$. Если все ЭВМ заняты, то задача направляется в очередь, которая на данный момент является минимальной;
	\item После выполнения в ЭВМ, задача возвращается на соответствующий терминал, причем 30\% задач обслуживается в АЦПУ $[b\pm\varepsilon]$ сек.
\end{enumerate}
\textbf{Цель:} Разработать GPSSV-модель для анализа процесса функционирования вычислительного центра в течение одного часа.\\
\
\textbf{Первоначальный перечень экспериментов:} $\lambda=0.2, \mu=12, a=2, \delta=1, b=12$.
